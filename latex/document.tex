\documentclass[twocolumn,a4,twoside]{article}
\usepackage{xeCJK} 
\usepackage{amsmath, amsthm}
\usepackage{listings,xcolor}
\usepackage{geometry} 
\usepackage{fontspec}
\usepackage{graphicx}
\usepackage{fancyhdr} 
\setsansfont{Source Code Pro} 
\setmonofont[Mapping={}]{Source Code Pro} 
\geometry{left=3cm,right=1cm,top=2cm,bottom=1cm} 
\setlength{\columnsep}{30pt}


\pagestyle{fancy}
\fancyhead[RO,LE]{Page \thepage}
\fancyhead[LO,RE]{China University of Petroleum}
\renewcommand{\headrulewidth}{0.4pt} 
\renewcommand{\footrulewidth}{0.4pt}
\cfoot{}

\lstset{
	language    = c++,
	numbers     = left,
	numberstyle = \tiny,
	breaklines  = true,
	captionpos  = b,
	tabsize     = 4,
	frame       = shadowbox,
	columns     = fullflexible,
	keywordstyle = \bfseries,
	basicstyle   = \small\ttfamily,
	stringstyle  = \color[RGB]{64,64,64}\ttfamily,
	rulesepcolor = \color{red!20!green!20!blue!20},
	showstringspaces = false,
}


\title{\CJKfamily{hei} {\bfseries ACM模板}}
\author{han777404}
\renewcommand{\today}{\number\year 年 \number\month 月 \number\day 日}

\begin{document}\small
	\pagestyle{headings}
	\begin{titlepage}
		\maketitle
	\end{titlepage}
	\newpage
	\pagestyle{empty}
	\renewcommand{\contentsname}{目录}
	\tableofcontents
	\newpage\clearpage
	\newpage
	\pagestyle{fancy}
	\setcounter{page}{1}   
	
	

	\section{数学}
		\subsection{欧拉素数筛}
		\lstinputlisting{../src/math/欧拉素数筛.cpp}
		\subsection{BSGS}
		\lstinputlisting{../src/math/BSGS.cpp}
		\subsection{Sqrt}
		\lstinputlisting{../src/math/Sqrt.java}
		\subsection{中国剩余定理}
		\lstinputlisting{../src/math/中国剩余定理.cpp}
		\subsection{快速幂}
		\lstinputlisting{../src/math/快速幂.cpp}
		\subsection{本原勾股数树}
		\lstinputlisting{../src/math/本原勾股数树.cpp}
		\subsection{欧拉函数phi}
		\lstinputlisting{../src/math/欧拉函数phi.cpp}
		\subsection{矩阵快速幂}
		\lstinputlisting{../src/math/矩阵快速幂.cpp}
		\subsection{组合数}
		\lstinputlisting{../src/math/组合数.cpp}
	
	\section{图论}
		\subsection{dijkstra}
		\lstinputlisting{../src/graph/dijkstra.cpp}
		\subsection{次短路}
		\lstinputlisting{../src/graph/次短路.cpp}
		\subsection{lca}
		\lstinputlisting{../src/graph/lca.cpp}
		\subsection{最大流}
		\lstinputlisting{../src/graph/最大流.cpp}
		\subsection{最小费用流}
		\lstinputlisting{../src/graph/最小费用流.cpp}
	
	\section{数据结构}
		\subsection{SegmentTree}
		\lstinputlisting{../src/datastructure/SegmentTree.cpp}
		\subsection{treap}
		\lstinputlisting{../src/datastructure/treap.cpp}
		\subsection{树状数组}
		\lstinputlisting{../src/datastructure/树状数组.cpp}
		\subsection{树链剖分}
		\lstinputlisting{../src/datastructure/树链剖分.cpp}
		\subsection{线段树}
		\lstinputlisting{../src/datastructure/线段树.cpp}
	
	
	\section{字符串}
		\subsection{AC自动机}
		\lstinputlisting{../src/string/AC自动机.cpp}
		\subsection{kmp}
		\lstinputlisting{../src/string/kmp.cpp}
		
	\section{其它}
		\subsection{FastIO}
		\lstinputlisting{../src/others/FastIO.cpp}
		\subsection{header}
		\lstinputlisting{../src/others/header.cpp}
		\subsection{凸包旋转卡壳}
		\lstinputlisting{../src/others/凸包旋转卡壳.cpp}
		\subsection{最远曼哈顿距离}
		\lstinputlisting{../src/others/最远曼哈顿距离.cpp}
		\subsection{最长上升子序列}
		\lstinputlisting{../src/others/最长上升子序列.cpp}


\end{document}
